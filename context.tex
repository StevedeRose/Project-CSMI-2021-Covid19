\documentclass{article} 
\usepackage[utf8]{inputenc}
\usepackage{natbib}
\usepackage{graphicx}
\usepackage[x11names]{xcolor}
\usepackage[english]{babel}
\usepackage{amsmath}
\usepackage{pgfgantt}

\title{Modelling Airborne Transmission of COVID-19}
\author{Mariam Grigoryan \and Anita Klein \and Steve de Rose}
\date{}

\begin{document}
\maketitle
\section*{Context}
Covid-19 virus was reported to the World Health Organization (WHO) on December 31, 2019, and was declared by the WHO as a global pandemic on March 11 2020. 
Since, avoiding the spread of this virus has been a priority. To reduce its spread, many countries have decided to isolate their population as much as possible, and have implemented social isolation measures that transformed people's lives.

In order to come back to our "normal lives" and remove all the social distancing measures, we must insure it can be done in a safe way and not put people in danger. 

Cemosis and Synapse-Concept are working together on a project called 4fastsim-ibat. The project aims to reduce and control energy consumption in buildings. Air is a major component in a building while trying to control energy consumption. Since Covid-19 has appeared in our lives, the question of air quality is crucial.

In fact, according to United States Environmental Protection Agency, the spread of COVID-19 may also occur via airborne particles in indoor environments, in some circumstances beyond the 2 m range encouraged by some social distancing recommendations.
Therefore, air circulation in a building has to be controlled to avoid contamination. 

In this project, we are going to work on modelling airborne transmission of COVID-19 in indoor spaces using an advection–diffusion–reaction equation. Then, we will associate this equation with the Navier-Stokes turbulnece model. And we will finally study the transmission in a whole building.

\section*{Cemosis}

Cemosis was created in January 2013 by Christophe Prud’homme and is hosted by IRMA (the Institute of Advanced Mathematical Research). Cemosis is the Strasbourg Centre for Modelling and Simulation. It offers expertise in Modeling Simulation and Optimization (MSO), Data Science, Big Data, Smart Data (DS), High Performance Computing, Parallel Computing, Cloud Computing (HPC) and in Signal and Image processing (SI). 

\section*{Synapse-Concept}

Synapse-Concept is a company specialised in engineering and technical studies. This company was created in November 1999.

\section*{Plan}

\begin{itemize}
    \item Project presentation
    \item Study the models
    \item Reproduction of the models
    \item Association with air models (if enough time available)
\end{itemize}
\pagebreak

\section*{Our Roadmap}
\begin{center}    
\begin{ganttchart}[
    x unit=0.8cm,
    y unit title=0.8cm,
    y unit chart=0.7cm,
    vgrid,
    hgrid,
    bar/.append style={fill=red!50},
    group/.append style={draw=black, fill=green!50},
    milestone/.append style={fill=orange, rounded corners=3pt}
    ]{1}{9}
    
        \gantttitle{2021}{9}\\
        \gantttitle{March/April}{5}
        \gantttitle{May}{4}\\
        \gantttitle{Week}{9}\\
        \gantttitlelist{1,...,9}{1} \\
        \ganttgroup{Overall target}{1}{9} \\ % elem0
        \ganttbar{Project's presentation}{1}{2} \\ % elem1
        \ganttbar{Collaboration with companies}{1}{2} \\ % elem2
        \ganttmilestone{V0}{3} \\ % elem3
        \ganttlink{elem1}{elem3}
        \ganttlink{elem2}{elem3}
    
        \ganttbar{1th sub-objective}{3}{4} \\ % elem4
        \ganttbar{2nd sub-objective}{3}{5} \\% elem5
        \ganttbar{Study of test cases}{5}{6} \\ % elem6
        \ganttbar{The output's Reproduction}{5}{7} \\ % elem7
        
        \ganttmilestone{V1}{7} \\ % elem8
        \ganttlink{elem5}{elem6}
        \ganttlink{elem6}{elem7}
        \ganttlink{elem7}{elem8}
    
        \ganttbar{Pre-final Release}{7}{8} \\ % elem9
        \ganttbar{Report}{7}{8} \\% elem10
        \ganttmilestone{V2}{8} \\% elem11
        
        \ganttlink{elem9}{elem10}
        \ganttlink{elem10}{elem11}
        
        \ganttbar{Final Outputs}{8}{9} \\ % elem12
        \ganttbar{Oral Defense}{9}{9}   \\ % elem13
        \ganttmilestone{Vfinal}{9} \\% elem14
        \ganttlink{elem12}{elem13}
        \ganttlink{elem13}{elem14}
        
        \ganttlink{elem3}{elem8}
        \ganttlink{elem8}{elem11}
        \ganttlink{elem11}{elem14}
\end{ganttchart}
\end{center}


\documentclass{article}
\usepackage[utf8]{inputenc}
\usepackage{natbib}
\usepackage{graphicx}

\begin{document}

\section*{Modelling Airborne Transmission of COVID-19}

Following the emergence and quick transmission of the Covid-19 virus, and knowing this virus is transmitted through virus-carrying respiratory droplets, we need a model for indoors airborne transmission that is quick to implement. \\
In order to create such a model, we make some assumptions:
\begin{itemize}
    \item the air in the room follows an advection-diffusion-reaction equation
    \item there is only one infectious person in the room
    \item the size of the room is 8m(l) ×8m(w) ×3m(h). 
\end{itemize}
And we test the effect on several ventilation settings:
\begin{itemize}
    \item breathing with and without a face mask
    \item talking with and without a face mask.
\end{itemize}

Models studying risk of airborne transmission generally fall into one of two types: the well-mixed room (WMR) assumption and Computational Fluid Dynamics (CFD) models.

Well Mixed Room model assume virus-carrying aerosols are evenly distributed throughout the room, so everyone in the room is equally likely to be infected. This assumption greatly simplifies the problem by ignoring the complex effects of the air flow on the airborne particles. These types of models are therefore quick to run, but also too simplistic.
Computational Fluid Dynamics simulations are useful in studying airborne transmission as they can take into account the details of the room size, geometry, the complex turbulent airflow and the size distribution of the aerosols. These models are accurate, but take a long time to run and are not suited for 'dynamic  simulations'.

Therefore, we develop a model that provides quick simulations and still takes into account turbulence airflow.
This model is based on the advection–diffusion–reaction equation.

\section*{Methodology}

We create a  2D  model  to  study  the  concentration  of  airborne  infectious  particles  in  a rectangular domain.
Respiratory droplets are transported via advection due to the airflow in the room and diffusion due to turbulent mixing. We assume the recirculation of air leads to turbulent mixing of the infectious particles.

Here is the governing equation for the airborne transmission of COVID-19, the advection–reaction–diffusion equation:

$$ \frac{\partial C}{\partial t} = \nabla . (K  \nabla  C) - \nabla . (\overrightarrow{v}  C) + S.$$
C = concentration of airborne infectious particles \\
t = time \\
$\nabla$ = two-dimensional gradient operator \\
K = isotropic eddy diffusion coefficient (turbulent diffusion ) \\
$\overrightarrow{v}$ = advection velocity of the air \\
S = sum of sources and sinks of viral particles.

We model the infected person breathing or talking as a continuous point source emitting viral particles at a constant rate of R particles per second. Thus, an infectious person talking or breathing at position (x0,y0) starting at time t = $t_0$ is modelled as follows:
$$ S_{inf} = R \ \delta(x-x_0) \ \delta(y-y_0) \ H(t-t_0). $$
$\delta$(x) is the Kronecker delta function and H(t) is the Heaviside step function.

Because R is unknown so far, we assume every respiratory droplets contains the virus and that inhaling and exhaling occur at the same rate.
We also assume there is mechanical ventilation in this indoor space provided by air vents.
$$S_{vent} = - \lambda C, $$

where $\lambda$ is the air exchange rate of the room. This is an approximation since removal of the air by ventilation occurs at a higher rate near the air vents.

Assuming there is only one infectious person, we obtain the following partial differential equation (PDE):

$$ \frac{\partial C}{\partial t} + \overrightarrow{v} . (\nabla C) - K  \nabla^2  C =
R \ \delta(x-x_0) \ \delta(y-y_0) \ H(t-t_0) - \lambda C.$$

\end{document}


\section*{Bibliography}
http://www.cemosis.fr/projects/4fastsim-ibat/

\end{document}

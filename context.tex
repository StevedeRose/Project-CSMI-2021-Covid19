\documentclass{article} 
\usepackage[utf8]{inputenc}
\usepackage{natbib}
\usepackage{graphicx}
\usepackage[x11names]{xcolor}
\usepackage[english]{babel}
\usepackage{amsmath}
\usepackage{pgfgantt}

\title{Modelling Airborne Transmission of COVID-19}
\author{Mariam Grigoryan \and Anita Klein \and Steve de Rose}
\date{}

\begin{document}
\maketitle
\section*{Context}
Covid-19 virus was reported to the World Health Organization (WHO) on December 31, 2019, and was declared by the WHO as a global pandemic on March 11 2020. 
Since, avoiding the spread of this virus has been a priority. To reduce its spread, many countries have decided to isolate their population as much as possible, and have implemented social isolation measures that transformed people's lives.

In order to come back to our "normal lives" and remove all the social distancing measures, we must insure it can be done in a safe way and not put people in danger. 

Cemosis and Synapse-Concept are working together on a project called 4fastsim-ibat. The project aims to reduce and control energy consumption in buildings. Air is a major component in a building while trying to control energy consumption. Since Covid-19 has appeared in our lives, the question of air quality is crucial.

In fact, according to United States Environmental Protection Agency, the spread of COVID-19 may also occur via airborne particles in indoor environments, in some circumstances beyond the 2 m range encouraged by some social distancing recommendations.
Therefore, air circulation in a building has to be controlled to avoid contamination. 

In this project, we are going to work on modelling airborne transmission of COVID-19 in indoor spaces using an advection–diffusion–reaction equation. Then, we will associate this equation with the Navier-Stokes turbulnece model. And we will finally study the transmission in a whole building.

\section*{Cemosis}

Cemosis was created in January 2013 by Christophe Prud’homme and is hosted by IRMA (the Institute of Advanced Mathematical Research). Cemosis is the Strasbourg Centre for Modelling and Simulation. It offers expertise in Modeling Simulation and Optimization (MSO), Data Science, Big Data, Smart Data (DS), High Performance Computing, Parallel Computing, Cloud Computing (HPC) and in Signal and Image processing (SI). 

\section*{Synapse-Concept}

Synapse-Concept is a company specialised in engineering and technical studies. This company was created in November 1999.

\section*{Plan}

\begin{itemize}
    \item Project presentation
    \item Study the models
    \item Reproduction of the models
    \item Association with air models (if enough time available)
\end{itemize}
\pagebreak

\section*{Our Roadmap}
\begin{center}    
\begin{ganttchart}[
    x unit=0.8cm,
    y unit title=0.8cm,
    y unit chart=0.7cm,
    vgrid,
    hgrid,
    bar/.append style={fill=red!50},
    group/.append style={draw=black, fill=green!50},
    milestone/.append style={fill=orange, rounded corners=3pt}
    ]{1}{9}
    
        \gantttitle{2021}{9}\\
        \gantttitle{March/April}{5}
        \gantttitle{May}{4}\\
        \gantttitle{Week}{9}\\
        \gantttitlelist{1,...,9}{1} \\
        \ganttgroup{Overall target}{1}{9} \\ % elem0
        \ganttbar{Project's presentation}{1}{2} \\ % elem1
        \ganttbar{Collaboration with companies}{1}{2} \\ % elem2
        \ganttmilestone{V0}{3} \\ % elem3
        \ganttlink{elem1}{elem3}
        \ganttlink{elem2}{elem3}
    
        \ganttbar{1th sub-objective}{3}{4} \\ % elem4
        \ganttbar{2nd sub-objective}{3}{5} \\% elem5
        \ganttbar{Study of test cases}{5}{6} \\ % elem6
        \ganttbar{The output's Reproduction}{5}{7} \\ % elem7
        
        \ganttmilestone{V1}{7} \\ % elem8
        \ganttlink{elem5}{elem6}
        \ganttlink{elem6}{elem7}
        \ganttlink{elem7}{elem8}
    
        \ganttbar{Pre-final Release}{7}{8} \\ % elem9
        \ganttbar{Report}{7}{8} \\% elem10
        \ganttmilestone{V2}{8} \\% elem11
        
        \ganttlink{elem9}{elem10}
        \ganttlink{elem10}{elem11}
        
        \ganttbar{Final Outputs}{8}{9} \\ % elem12
        \ganttbar{Oral Defense}{9}{9}   \\ % elem13
        \ganttmilestone{Vfinal}{9} \\% elem14
        \ganttlink{elem12}{elem13}
        \ganttlink{elem13}{elem14}
        
        \ganttlink{elem3}{elem8}
        \ganttlink{elem8}{elem11}
        \ganttlink{elem11}{elem14}
\end{ganttchart}
\end{center}


\section*{Bibliography}
http://www.cemosis.fr/projects/4fastsim-ibat/

\end{document}

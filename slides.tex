\documentclass[10pt]{beamer}
\usetheme{Marburg}
\graphicspath{{Arquivos/}}
\usepackage[utf8]{inputenc}
\usepackage[french]{babel}
\usepackage[T1]{fontenc}
\usepackage{amsmath}
\usepackage{amsfonts}
\usepackage{amssymb}
\usepackage{graphicx}
\author{Anita Klein - Mariam Grigoryan - Steve de Rose}
\newcommand{\TT}{}
\newcommand{\PI}{}
\newcommand{\PII}{}
\newcommand{\PIII}{}
\newcommand{\PIV}{}
\newcommand{\PV}{}
\newcommand{\PVI}{}
\title{Airborne Transmission of Covid-19}


\begin{document}
    
\begin{frame}

    \maketitle

\end{frame}


\section{\PI} 
\begin{frame}{The Concept}{\PI} 
    \begin{itemize}
        \item Covid 19 virus reported to the World Health Organization (WHO) on December 31, 2019. 
        \item Reduce/Prevent its spread
        \item Cemosis and Synapse-Concept project 4fastsim-ibat. 
        \item Air quality since Covid-19.
    \end{itemize}  

\end{frame}

\section{\PII} 
\begin{frame}{Collaboration Cemosis/Synapse}{\PII} 
    \begin{itemize}
        \item Cemosis created in January 2013 by Christophe Prud’homme. 
        \item Strasbourg Centre for Modelling and Simulation.
        \item Synapse-Concept created in November 1999.
        \item Specialised in engineering and technical studies.
    \end{itemize}  

\end{frame}

\section{\PIII} 
\begin{frame}{Project}{\PIII} 
    \begin{itemize}
        \item Study of the airborne transmission of COVID-19 in an indoor space.
        \item The air in the room follows an advection-diffusion-reaction equation.
        \item With only one infectious person in the room.
        \item Rome of size $8m(l)\times 8m(w) \times 3m(h)$.
        \item Breathing/Talking with and without a face mask.
    \end{itemize}  

\end{frame}

\section{\PIV} 
\begin{frame}{Methodology}{\PIV} 
    \begin{itemize}
        \item Firstly, 2D model to study/reproduce the concentration of airborne infectious particles.
        \item Using the advection–reaction–diffusion equation.
        \item Secondly, using «N-point ASOM» (air supply opening model).
    \end{itemize}  

\end{frame}

\section{\PV} 
\begin{frame}{The Tools}{\PV} 
    \begin{itemize}
        \item Feel++ to solve advection–reaction–diffusion equation.
        \item Paraview to visualize the solution.
        \item Antora to generate the documentation site.
        \item Visual Studio Code.
    \end{itemize}  

\end{frame}

\section{\PVI} 
\begin{frame}{References }{\PVI} 
    \begin{itemize}
        \item Z. Lau, K. Kaouri, I. Griffiths. Modelling Airborne Transmission of COVID-19 in Indoor Spaces Using anAdvection–Diffusion–Reaction Equation.School of Mathematics, Cardiff University and Mathematical Institute, University of Oxford.
        \item B. Zhao and X. Li. A simplified system for indoor airflow simulation. Building and Environment · April 2003
        \item Zohra Djatouti, Christophe Prud’homme, Vincent Chabannes, Romain Hild IBat Website 
    \end{itemize}  

\end{frame}







\end{document}